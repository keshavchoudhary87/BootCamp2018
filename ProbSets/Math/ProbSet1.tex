\documentclass{article}
\usepackage{amsmath}
\usepackage{amsfonts}

\title{Maths Problem Set- Measure Theory}
\author{Keshav Choudhary}

\begin{document}
  \maketitle

  \section*{Problem 1}
    \begin{itemize}

      \item Consider $A \in \mathcal{G}_1 \implies A$ open on $\mathbb{R}\implies {A}^c$
      is either closed on $\mathbb{R}$ or semi-open on $\mathbb{R}$. So ${A}^c \notin \mathcal{G}_1$
      as it is not a purely open interval. Hence $\mathcal{G}_1$ is not a $\sigma$- algebra nor an algebra.

      \item Consider $A_n \in \mathcal{G}_2, n \in \mathbb{N}$. Then $\bigcup_{n=1}^{\infty}A_n \notin \mathcal{G}_2$
      since $\mathcal{G}_2$ contains only sets which are finite unions of intervals of the form $(a, b], (-\infty, b], (a, \infty)$.
      Thus $\mathcal{G}_2$ is not a $\sigma$- algebra. Now we check whether $\mathcal{G}_2$  is an algebra.
      It is clear that $\phi \in \mathcal{G}_2$. Now consider any interval of the form $(a,b]$.
      Then it's complement is of the form $(-\infty,a] \cup (b, \infty)$ which $\in \mathcal{G}_2$. Similarly for any interval of the form
      $(-\infty, b]$, its complement is of the form $(b, \infty)$ which $\in \mathcal{G}_2$. Thus, for all $A \in \mathcal{G}_2$, ${A}^c \in \mathcal{G}_2$

      $\bigcup_{n=1}^{\mathbb{N}}A_{n}$    

      \item


    \end{itemize}






\end{document}
