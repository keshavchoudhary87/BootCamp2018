\documentclass[letterpaper,12pt]{article}
\usepackage{array}
\usepackage{threeparttable}
\usepackage{geometry}
\usepackage{bm}
\geometry{letterpaper,tmargin=1in,bmargin=1in,lmargin=1.25in,rmargin=1.25in}
\usepackage{fancyhdr,lastpage}
\pagestyle{fancy}
\lhead{}
\chead{}
\rhead{}
\lfoot{}
\cfoot{}
\rfoot{\footnotesize\textsl{Page \thepage\ of \pageref{LastPage}}}
\renewcommand\headrulewidth{0pt}
\renewcommand\footrulewidth{0pt}
\usepackage[format=hang,font=normalsize,labelfont=bf]{caption}
\usepackage{listings}
\lstset{frame=single,
  language=Python,
  showstringspaces=false,
  columns=flexible,
  basicstyle={\small\ttfamily},
  numbers=none,
  breaklines=true,
  breakatwhitespace=true
  tabsize=3
}
\usepackage{amsmath}
\usepackage{amssymb}
\usepackage{amsthm}
\usepackage{harvard}
\usepackage{setspace}
\usepackage{float,color}
\usepackage[pdftex]{graphicx}
\usepackage{hyperref}
\hypersetup{colorlinks,linkcolor=red,urlcolor=blue}
\theoremstyle{definition}
\newtheorem{theorem}{Theorem}
\newtheorem{acknowledgement}[theorem]{Acknowledgement}
\newtheorem{algorithm}[theorem]{Algorithm}
\newtheorem{axiom}[theorem]{Axiom}
\newtheorem{case}[theorem]{Case}
\newtheorem{claim}[theorem]{Claim}
\newtheorem{conclusion}[theorem]{Conclusion}
\newtheorem{condition}[theorem]{Condition}
\newtheorem{conjecture}[theorem]{Conjecture}
\newtheorem{corollary}[theorem]{Corollary}
\newtheorem{criterion}[theorem]{Criterion}
\newtheorem{definition}[theorem]{Definition}
\newtheorem{derivation}{Derivation} % Number derivations on their own
\newtheorem{example}[theorem]{Example}
\newtheorem{exercise}[theorem]{Exercise}
\newtheorem{lemma}[theorem]{Lemma}
\newtheorem{notation}[theorem]{Notation}
\newtheorem{problem}[theorem]{Problem}
\newtheorem{proposition}{Proposition} % Number propositions on their own
\newtheorem{remark}[theorem]{Remark}
\newtheorem{solution}[theorem]{Solution}
\newtheorem{summary}[theorem]{Summary}
%\numberwithin{equation}{section}
\bibliographystyle{aer}
\newcommand\ve{\varepsilon}
\newcommand\boldline{\arrayrulewidth{1pt}\hline}


\begin{document}

\begin{flushleft}
  \textbf{\large{Maths Problem Set \# 5}} \\
  Keshav Choudhary \footnote{I have worked with Navneeraj Sharma and Shekhar Kumar on this problem set}
\end{flushleft}

\vspace{5mm}

\noindent\textbf{Problem 8.1} See Jupyter Notebook for the plot and solution.

\vspace{2mm}

\noindent\textbf{Problem 8.2} See Jupyter Notebook for the plot and solution.

\vspace{2mm}

\noindent\textbf{Problem 8.3}
Let $x$ be the quantity of production of GI Bard Soldiers and let $y$ be the quantity of production of Joey dolls.
The revenues are given by $12x + 10y$, the raw material cost is given by $5x+3y$ and the overhead costs are given by
$F+3x+4y$ where $F$ is the overhead cost when there is no production.

The finishing labour requirement is $15x+10y$ minutes and molding labour requirement is $2x+2y$ minutes.

The optimization problem is thus:
\begin{align*}
  \max_{{x,y}}4x+3y\\
  \textnormal{subject to: } 3x+2y &\le 360\\
   x+y &\le 150\\
   y &\le 200
\end{align*}

\noindent\textbf{Problem 8.4}
The optimization problem is as follows-
\begin{align*}
  \min_{x_{i,j}} & \bigl( 5x_{AD} + 2x_{AB} + 2x_{BD} + 7x_{BE} + 9x_{BF} + 5x_{BC}\\
                 & 2x_{CF} + 4x_{DE}+ 3x_{EF}\bigr)\\
  \textnormal{subject to: }\\
  &x_{AD} + x_{AB} = 10\\
  &x_{BC} + x_{BD} + x_{BE} + x_{BF} - x_{AB} = 1\\
  &x_{CF} - x_{BC} = -2\\
  &x_{DE} - x_{AD} - x_{BD} = -3\\
  &x_{EF} - x_{BE} - x_{BD} = 4 \\
  &x_{CF} + x_{BF} + x_{EF} = 10\\
  & 0 \le x_{i,j} \le 6 \textnormal{ where $i,j$ are nodes}
\end{align*}

\noindent\textbf{Problem 8.5}


\textbf{(i)}The initial dictionary after adding in 3 slack variables $x_3, x_4, x_5$ is:
  \begin{align*}
      \zeta_1 &= 3x_1+x_2\\
      \cline{1-2}
      x_3 &= 15 - x_1 - 3x_2\\
      x_4 &= 18-2x_1-3x_2\\
      x_5 &= 4-x_1+x_2
  \end{align*}
We choose $x_1$ as the entering variable and $x_5$ as the leaving variable. The new dictionary becomes:

\begin{align*}
    \zeta_2 &= 12+4x_2-3x_5\\
    \cline{1-2}
    x_1 &= 4 + x_2 - x_5\\
    x_3 &= 11-4x_2+ x_5\\
    x_4 &= 10-5x_2+2x_5\\
\end{align*}

We now choose $x_2$ as the entering variable and $x_4$ as the leaving variable. The new dictionary becomes:

\begin{align*}
    \zeta_3 &= 20 - (4/5)x_4 - (7/5)x_5\\
    \cline{1-2}
    x_1 &= 6 - (1/5)x_4 - (3/5)x_5\\
    x_2 &= 2-(1/5)x_4+ (2/5)x_5\\
    x_3 &= 3 + (4/5)x_4 -(13/5)x_5\\
\end{align*}

Since both $x_4, x_5$ now appear with negative signs in the objective function, this is the optimum. The
values are: $x_1= 6, x_2=2$ and the value of the objective function is $20$. This matches the answer in the
Jupyter Notebook.

\textbf{(ii)} The initial dictionary after adding in 3 slack variables $x_3, x_4, x_5$ is:
\begin{align*}
      \zeta_1 &= 4x_1+6x_2\\
      \cline{1-2}
      x_3 &= 11 + x_1 - x_2\\
      x_4 &= 27-x_1-x_2\\
      x_5 &= 90 - 2x_1 - 5x_2\\
\end{align*}

We choose $x_1$ as the entering variable and $x_4$ as the leaving variable. The new dictionary becomes:

\begin{align*}
  \zeta_2 &= 108 + 2x_2 -4x_4\\
  \cline{1-2}
  x_1 &= 27 - x_2 - x_4\\
  x_3 &= 38- 2x_2-x_4\\
  x_5 &= 36 - 3x_2 + 2x_4\\
\end{align*}

We now choose $x_2$ as the entering variable and $x_5$ as the leaving variable. The new dcitionary becomes:
\begin{align*}
  \zeta_3 &= 132 -(8/3)x_4-(2/3)x_5\\
  \cline{1-2}
  x_1 &= 15-(5/3)x_4+(1/3)x_5\\
  x_2 &= 12+(2/3)x_4-(1/3)x_5\\
  x_3 &= 14-3x_4+(2/3)x_5 \\
\end{align*}

All the variables now appear in the objective function with a negative sign. Hence the present choice is optimal.
This occurs at $x=15, y=12$ and the value of the objective function is $132$.

\vspace{2mm}
\noindent\textbf{Problem 8.6} After adding 3 slack variables $w_1, w_2, w_3$ , the initial dictionary is:

\begin{align*}
  \zeta_1 &= 4x+3y\\
  \cline{1-2}
  w_1 &= 360-3x-2y\\
  w_2&= 150-x-y\\
  w_3 &= 200-y \\
\end{align*}
We choose as x as the entering variable and $w_1$ as the leaving variable. The new dictionary becomes:

\begin{align*}
  \zeta_2 &= 480 + (1/3)y -(4/3)w_1\\
  \cline{1-2}
  x &= 120 -(2/3)y - (1/3)w_1\\
  w_2&= 30-(1/3)y+(1/3)w_1\\
  w_3 &= 200-y \\
\end{align*}

Next, we choose y as the entering variable and $w_2$ as the leaving variable.

\begin{align*}
  \zeta_3 &= 510 -10w_2 - (5/3)w_1\\
  \cline{1-2}
  x &= 60 + 20w_2+(1/3)w_1\\
  y &= 90 - 30w_2-w_1\\
  w_3 &= 110+30w_2+w_1 \\
\end{align*}

As all the terms in the objective function appear with a negative sign, we are at the optimum.
The value of the objective function i.e profit is \textdollar 510 and $x=60, y=90$

\vspace{3mm}
\noindent\textbf{Problem 8.7}
\begin{enumerate}
  \item The origin is not part of the feasible set. This can be seen from the Jupyter Notebook where the
  feasible set is plotted. We therefore set up an auxiliary problem first by subtracting $x_0$ from all the
  constraints. The dictionary for the auxiliary problem is:
  \begin{align*}
    \zeta_1 &= -x_0\\
    \cline{1-2}
    x_3 &= -8 + 4x_1 + 2x_2 + x_0\\
    x_4 &= 6 + 2x_1 -3x_2+ x_0\\
    x_5 &= 3-x_1+x_0
  \end{align*}

  We pivot $x_0$ and $x_1$. The new dictionary becomes:
  \begin{align*}
    \zeta_2 &= -x_0\\
    \cline{1-2}
    x_1 &= 2 - (1/2)x_2 + (1/4)x_3-(1/4)x_0\\
    x_4 &= 10 - 4x_2 +(1/2)x_3+ (1/2)x_0\\
    x_5 &= 1+(5/2)x_2 - (1/4)x_3 + (5/4)x_0\\
  \end{align*}
  We can see that this dictionary is optimal as all points are feasible and the objective function is 0
  Thus $x_1=2, x_2=0, $ is a feasible point for the original problem. We can remove $x_0$ from the main problem
  and replace $x_1$ in terms of the non-basic variables. The new dictionary becomes:

  \begin{align*}
    \zeta_3 &= 2 -(1/2)x_2+2x_2+(1/4)x_3\\
    \cline{1-2}
    x_1 &= 2 - (1/2)x_2 + (1/4)x_3\\
    x_4 &= 10 - 4x_2 +(1/2)x_3\\
    x_5 &= 1 + (1/2)x_2 - (1/4)x_3\\
  \end{align*}
  We pivot $x_3, x_5$. The new dictionary becomes:
  \begin{align*}
    \zeta_4 &= 3+2x_2-x_5\\
    \cline{1-2}
    x_1 &= 3 - x_5\\
    x_3 &= 4+2x_2-4x_5\\
    x_4 &=12-3x_2-2x_5\\
  \end{align*}
We again pivot $x_2, x_4$ and obtain the dictionary:
\begin{align*}
  \zeta_5 &= 11- (2/3)x_4- (7/3)x_5\\
  \cline{1-2}
  x_1 &= 3-x_5\\
  x_2 &= 4 - (1/3)x_4 - (2/3)x_5\\
  x_3 &= 12 - (2/3)x_4 - (16/3)x_5
\end{align*}
This dictionary is optimal as all the terms appear with a negative sign. The optimal values are $x_1=3, x_2=4$.
This can also be confirmed from the diagram in the Jupyter Notebook.

\item The origin is not part of the feasible set as the third constraint appears with a negative sign.
The auxiliary problem is:
\begin{align*}
  \zeta_1 &= -x_0\\
  \cline{1-2}
  x_3 &= 15-5x_1-3x_2+x_0\\
  x_4 &= 15-3x_1-5x_2+x_0\\
  x_5 &= -12 -4x_1+3x_2+x_0\\
\end{align*}

We pivot $x_0, x_5$ and get the new dictionary as:
\begin{align*}
  \zeta_2 &= 12+4x_1-3x_2+x_5\\
  \cline{1-2}
  x_0 &= 12+4x_1-3x_2+x_5\\
  x_3 &= 27-x_1 - 6x_2+x_5\\
  x_4 &= 27+x_1- 8x_2+x_5\\
\end{align*}

Again, we pivot $x_2, x_4$ and obtain:
\begin{align*}
  \zeta_3 &= -(15/8)- (29/8)x_1-(3/8)x_4- (5/8)x_5\\
  \cline{1-2}
  x_3 &= (27/4) - (7/4)x_1+(3/4)x_4+(1/4)x_5\\
  x_2 &= (27/8) + (1/8)x_1 -(1/8)x_4+(1/8)x_5\\
  x_0 &= (15/8) + (29/8)x_1 + (3/8)x_4 + (5/8)x_5\\
\end{align*}

This dictionary is optimal since all the coefficients in the objective function are negative. However,
at this optimum, $x_0 \ne 0$. Hence the problem is infeasible.

\item After adding in the slack variables, the initial dictionary is :
\begin{align*}
  \zeta_1 &= -3x_1+x_2\\
  \cline{1-2}
  x_3 &= 4-x_2\\
  x_4 &= 6+2x_1-3x_2\\
\end{align*}
We pivot, $x_2, x_4$ and obtain the following dictionary:
\begin{align*}
  \zeta_2 &= 2-(7/3)x_1-(1/3)x_4\\
  \cline{1-2}
  x_2 &= 2+(2/3)x_1 - (1/3)x_4\\
  x_3 &= 2-(2/3)x_1+(1/3)x_3\\
\end{align*}

This dictionary is optimal as all the coefficients in the objective function have negative signs.
The optimal solution is $x_1= 0, x_2 = 2$ and the value of the objective function is 2.
\end{enumerate}

\vspace{3mm}
\noindent\textbf{Problem 8.12}
The initial dictionary after adding in the slack variables is:
\begin{align*}
  \zeta_1 &= 10x_1-57x_2-9x_3-24x_4\\
  \cline{1-2}
  x_5 &= -0.5x_1 + 1.5x_2+0.5x_3-x_4\\
  x_6 &= -0.5x_1+5.5x_2 + 2.5x_3 -9x_4\\
  x_7 &= 1-x_1\\
\end{align*}

Using Bland's rule, we pivot $x_1, x_5$. The new dictionary is:
\begin{align*}
  \zeta_2 &= -27x_2+x_3-44x_4 - 20x_5\\
  \cline{1-2}
  x_1 &= 3x_2 + x_3 -2x_4-2x_5\\
  x_6 &= 4x_2+2x_3-8x_4+x_5\\
  x_7 &= 1-3x_2-x_3+2x_4+2x_5\\
\end{align*}

We now pivot $x_3, x_7$ and obtain:

\begin{align*}
  \zeta_3 &= 1-30x_2-42x_4-18x_5-x_7\\
  \cline{1-2}
  x_1 &= 1-x_7 \\
  x_6 &= 2-2x_2-4x_4+5x_5-2x_7 \\
  x_3 &= 1-3x_2+2x_4+2x_5-x_7 \\
\end{align*}

This dictionary is optimal as all the coefficients appear with negative sign in the objective function. The
optimal points are $x_1=1, x_2=0, x_3=1, x_4=0$ and the value of the objective function is 1.

\vspace{3mm}
\noindent\textbf{Problem 8.15}
  Using the definitions of primal and dual problems where $\bm{x, y}$ are feasible points of the primal
  and the dual respectively, we have
\begin{align*}
   \bm{A^Ty} &\succeq \bm{c} \\
   \Rightarrow \bm{x^TA^Ty} &\ge \bm {x^Tc}\\
   \Rightarrow \bm{(Ax)^Ty} &\ge \bm{(x^Tc)}\\
   \Rightarrow \bm{b^Ty} &\ge \bm{(Ax)^Ty} \ge \bm{x^Tc}\\
   \Rightarrow \bm{b^Ty} &\ge \bm{x^Tc}\\
   &=\bm{c^Tx}\\
   \Rightarrow \bm{b^Ty} &\ge \bm{c^Tx}
\end{align*}


\vspace{3mm}
\noindent\textbf{Problem 8.17}

Consider the primal problem
\begin{align*}
\max &\mathbf{c^Tx} \\
\textnormal{subject to } &A\textbf{x}\preceq \textbf{b}\\
&\textbf{x} \succeq 0\\
\end{align*}

The dual of the problem is
\begin{align*}
\min &\mathbf{b^Ty} \\
\textnormal{subject to } &A^T\textbf{y}\succeq \textbf{c}\\
&\textbf{y} \succeq 0\\
\end{align*}

The dual problem can be re-written as
\begin{align*}
\max (&\mathbf{-b^Ty}) \\
\textnormal{subject to } &-A^T\textbf{y}\preceq \textbf{-c}\\
&\textbf{y} \succeq 0\\
\end{align*}

Let us rename $$-b^T=r^T, -A^T = K, -c = p$$

The dual problem thus becomes-
\begin{align*}
\max (&\mathbf{r^Ty}) \\
\textnormal{subject to } &K\textbf{y}\preceq \textbf{p}\\
&\textbf{y} \succeq 0\\
\end{align*}

This is in the same form as a linear optimization problem. The dual of this problem is:
\begin{align*}
\min &\mathbf{p^Tx} \\
\textnormal{subject to } &K^T\textbf{x}\succeq \textbf{r}\\
&\textbf{x} \succeq 0\\
\end{align*}

On replacing the terms defined above we get:
\begin{align*}
\max &\mathbf{c^Tx} \\
\textnormal{subject to } &A\textbf{x}\preceq \textbf{b}\\
&\textbf{x} \succeq 0\\
\end{align*}

Which is nothing but the primal problem. Hence the dual of the dual is the primal problem.

\vspace{3mm}
\noindent\textbf{Problem 8.18}
We first solve the primal problem using the simplex method. The initial dictionary is

\begin{align*}
  \zeta_1 &= x_1+x_2\\
  \cline{1-2}
  x_3 &= 3-2x_1-x_2\\
  x_4 &= 5-x_1-3x_2\\
  x_5 &= 4-2x_1-3x_2\\
\end{align*}

We pivot $x_1, x_3$ and obtain the dictionary:
\begin{align*}
  \zeta_2 &= 1.5 + 0.5x_2-0.5x_3\\
  \cline{1-2}
  x_1 &= 1.5 -0.5x_2-0.5x_3\\
  x_4 &= 3.5-2.5x_2+0.5x_3\\
  x_5 &= 1-2x_2+x_3\\
\end{align*}

We now pivot $x_2,x_5$ and obtain:
\begin{align*}
  \zeta_3 &= 1.75 - 0.25x_3-0.25x_5\\
  \cline{1-2}
  x_1 &= 1.25 - 0.75x_3+0.25x_5\\
  x_4 &= 2.25 - 0.75x_3+1.25x_5\\
  x_2 &= 0.5+0.5x_3-0.5x_5\\
\end{align*}

This dictionary is optimal as the coefficients of the variables in the objective function appear with a
negative sign. The optimal values are $x_1=1.25, x_2 = 0.5$ and the objective function is 1.75.






\end{document}
