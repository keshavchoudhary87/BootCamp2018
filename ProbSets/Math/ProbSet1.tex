\documentclass{article}
\usepackage{amsmath}
\usepackage{amsfonts}

\title{Maths Problem Set- Measure Theory}
\author{Keshav Choudhary}

\begin{document}
  \maketitle

  \section*{Exercise 1.3}
    \begin{itemize}

      \item Consider $A \in \mathcal{G}_1 \implies A$ open on $\mathbb{R}\implies {A}^c$
      is either closed on $\mathbb{R}$ or semi-open on $\mathbb{R}$. So ${A}^c \notin \mathcal{G}_1$
      as it is not a purely open interval. Hence $\mathcal{G}_1$ is not a $\sigma$- algebra nor an algebra.

      \item Consider $A_n \in \mathcal{G}_2, n \in \mathbb{N}$. Then $\bigcup_{n=1}^{\infty}A_n \notin \mathcal{G}_2$
      since $\mathcal{G}_2$ contains only sets which are finite unions of intervals of the form $(a, b], (-\infty, b], (a, \infty)$.
      Thus $\mathcal{G}_2$ is not a $\sigma$- algebra. Now we check whether $\mathcal{G}_2$  is an algebra.
      It is clear that $\phi \in \mathcal{G}_2$. Now consider any interval of the form $(a,b]$.
      Then it's complement is of the form $(-\infty,a] \cup (b, \infty)$ which $\in \mathcal{G}_2$. Similarly for any interval of the form
      $(-\infty, b]$, its complement is of the form $(b, \infty)$ which $\in \mathcal{G}_2$. Thus, for all $A \in \mathcal{G}_2$, ${A}^c \in \mathcal{G}_2$.
      Now consider ${A}_n \in \mathcal{G}_2$ for $n \in \mathbb{N}$. Then $\bigcup_{n=1}^{N}A_{n}$ is also a finite union
      of disjoint intervals of the form $(-\infty, b], (a,b]$ and  $(a, \infty)$. Hence $\mathcal{G}_2$ is an algebra (but not a $\sigma$-algebra).

      \item Now consider $A_n \in \mathcal{G}_3, n \in \mathbb{N}$. The first two properties of an algebra hold in this case
      as they have already been proved above. Now consider $\bigcup_{n=1}^{\infty}A_{n}$ where ${A}_n \in \mathcal{G}_3$ for $n \in \mathbb{N}$.
      The countable union $\bigcup_{n=1}^{\infty}A_{n} \in \mathcal{G}_3$ as it is contains countable unions of intervals of the form
      $(a,b], (-\infty, b]$ and $(a, \infty)$. Thus $\mathcal{G}_3$ is a $\sigma$- algebra.

    \end{itemize}

  \section*{Exercise 1.7}
    Let $\mathcal{A}$ be any $\sigma$-algebra. By defnition of $\sigma$-algebra, $\phi \in \mathcal{A}$. Similarly, $X = {\phi}^c \in \mathcal{A}$
    Thus $\{\phi, X\} \subset \mathcal{A}$. Now consider any $A \in \mathcal{A}$. Since $\mathcal{A}$ is a $\sigma$-algebra on $X$, $A \subset X \Rightarrow
    A \in \mathcal{P}(X)$. Thus $\mathcal{A} \subset \mathcal{P}(X)$.

  \section*{Exercise 1.10}
  








\end{document}
