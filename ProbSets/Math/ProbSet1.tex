\documentclass{article}
\usepackage{amsmath}
\usepackage{amsfonts}

\title{Maths Problem Set- Measure Theory}
\author{Keshav Choudhary}
\date{}

\begin{document}
  \maketitle

  \section*{Exercise 1.3}
    \begin{itemize}

      \item Consider $A \in \mathcal{G}_1 \implies A$ open on $\mathbb{R}\implies {A}^c$
      is either closed on $\mathbb{R}$ or semi-open on $\mathbb{R}$. So ${A}^c \notin \mathcal{G}_1$
      as it is not a purely open interval. Hence $\mathcal{G}_1$ is not a $\sigma$- algebra nor an algebra.

      \item Consider $A_n \in \mathcal{G}_2, n \in \mathbb{N}$. Then $\bigcup_{n=1}^{\infty}A_n \notin \mathcal{G}_2$
      since $\mathcal{G}_2$ contains only sets which are finite unions of intervals of the form $(a, b], (-\infty, b], (a, \infty)$.
      Thus $\mathcal{G}_2$ is not a $\sigma$- algebra. Now we check whether $\mathcal{G}_2$  is an algebra.
      It is clear that $\phi \in \mathcal{G}_2$. Now consider any interval of the form $(a,b]$.
      Then it's complement is of the form $(-\infty,a] \cup (b, \infty)$ which $\in \mathcal{G}_2$. Similarly for any interval of the form
      $(-\infty, b]$, its complement is of the form $(b, \infty)$ which $\in \mathcal{G}_2$. Thus, for all $A \in \mathcal{G}_2$, ${A}^c \in \mathcal{G}_2$.
      Now consider ${A}_n \in \mathcal{G}_2$ for $n \in \mathbb{N}$. Then $\bigcup_{n=1}^{N}A_{n}$ is also a finite union
      of disjoint intervals of the form $(-\infty, b], (a,b]$ and  $(a, \infty)$. Hence $\mathcal{G}_2$ is an algebra (but not a $\sigma$-algebra).

      \item Now consider $A_n \in \mathcal{G}_3, n \in \mathbb{N}$. The first two properties of an algebra hold in this case
      as they have already been proved above. Now consider $\bigcup_{n=1}^{\infty}A_{n}$ where ${A}_n \in \mathcal{G}_3$ for $n \in \mathbb{N}$.
      The countable union $\bigcup_{n=1}^{\infty}A_{n} \in \mathcal{G}_3$ as it contains countable unions of intervals of the form
      $(a,b], (-\infty, b]$ and $(a, \infty)$. Thus $\mathcal{G}_3$ is a $\sigma$- algebra.

    \end{itemize}

  \section*{Exercise 1.7}
    Let $\mathcal{A}$ be any $\sigma$-algebra. By defnition of $\sigma$-algebra, $\phi \in \mathcal{A}$. Similarly, $X = {\phi}^c \in \mathcal{A}$
    Thus $\{\phi, X\} \subset \mathcal{A}$. Now consider any $A \in \mathcal{A}$. Since $\mathcal{A}$ is a $\sigma$-algebra on $X$, $A \subset X \Rightarrow
    A \in \mathcal{P}(X)$. Thus $\mathcal{A} \subset \mathcal{P}(X)$. Thus $\{\phi, X\} \subset A \subset \mathcal{P}(X)$

  \section*{Exercise 1.10}
    Since $\{S_\alpha\}$ is a family of $\sigma$- algebras, then $\phi \in S_\alpha \forall \alpha \Rightarrow \phi \in
    \bigcap_{\alpha}S_\alpha$. Now consider any $A \in  \bigcap_{\alpha}S_\alpha$. This means that $A \in S_\alpha$ for each $\alpha$
    Since each $S_\alpha$ is a $\sigma$ - algebra $\Rightarrow A^c \in S_\alpha \forall \alpha \Rightarrow A^c \in \bigcap_{\alpha}S_\alpha$.
    Now consider $\{A_n\} \in \bigcap_{\alpha}(S_\alpha) \forall n \in \mathbb{N}$. Then $A_n \in S_\alpha \forall \alpha, n \in \mathbb{N}$
    since each $S_\alpha$ is a $\sigma$-algebra, it means that $\bigcup_{n=1}^{\infty}A_n \in S_\alpha \forall \alpha$. This in turn means that
    $\bigcup_{n=1}^{\infty}A_n \in \bigcap_{\alpha}S_\alpha$. Hence $\bigcap_{\alpha}S_\alpha$ is a $\sigma$- algebra.

  \section*{Exercise 1.17}
    In order to prove the results we first prove a simpler result. We prove that if $\mu: \mathcal{S} \rightarrow [0, \infty]$ is a measure, then
    $\mu(\bigcup_{i=1}^{n=N}A_i) = \sum_{i=1}^{i=N}\mu(A_i)$ if $A_i \bigcap A_j = \phi, i \ne j$. To prove this, let all $A_i$ for $i > N = \phi$.
    Since $\mu(\phi)= 0$, we then get $\mu(\bigcup_{i=1}^{n=N}A_i) = \mu(\bigcup_{i=1}^{n=\infty}A_i) = \sum_{i=1}^{i=\infty}\mu(A_i) = \sum_{i=1}^{i=N}\mu(A_i)$.

    Now to prove monotonicity, consider two sets $A, B \in \mathcal{S}, A \subset B$. Now define $C = A^c \cap B$. Since $\mathcal{S}$ is a $\sigma$- algebra
    $A^c \cap B \in \mathcal{S}$. Furthermore, $A \cap C = \phi$. Since $\mu$ is a measure, $\mu(A \cup C) = \mu(A) + \mu(B) \Rightarrow
    \mu(B) = \mu(A) + \mu(C)$. Since the range of $\mu$ is non-negative, $\mu(C) \ge 0$. Thus $\mu(B) \ge \mu(A)$.

    Now we prove countable sub-additivity. Consider 2 sets $A_1, A_2$. We can write, $A_1 \cup A_2 = ({A_1}^c \cap A_2) \cup ({A_2}^c \cap A_1) \cup (A_1 \cap A_2)$
    i.e as a union of disjoint sets. Using the result proved above, we get $\mu(A_1 \cup A_2) = \mu({A_1}^c \cap A_2) + \mu({A_2}^c \cap A_1) + \mu(A_1 \cap A_2)
    \le \mu({A_1}^c \cap A_2) + \mu(A_1 \cap A_2) + \mu({A_2}^c \cap A_1) + \mu(A_1 \cap A_2) = \mu(A_1) + \mu(A_2)$. Thus we have $\mu (A_1 \cup A_2) \le \mu(A_1) + \mu(A_2)$
    The same argument can be carried out inductively for all $n \in \mathbb{N}$. For example in the case of three sets, $A_1, A_2$ and $A_3$, we can assume $A_1 \cup A_2 = A$
    and proceed as before. Therefore $\mu(\bigcup_{i=1}^{i=\infty}A_i) \le \sum_{i=1}^{i=\infty}\mu(A_i)$.

  \section*{Exercise 1.18}
    $\lambda(\phi) = \mu(\phi \cap B) = \mu(\phi) = 0$. Let $\{A_i\}_{i=1}^{i= \infty}$ be a collection of disjoint sets.
    We have, $\lambda(\bigcup_{i=1}^{i = \infty}A_i) = \mu(B \cap \bigcup_{i=1}^{i = \infty}A_i) = \mu(\bigcup_{i=1}^{i=\infty}(B \cap A_i))$ where we have used De-Morgan's laws in the last step.
    Since all $A_i$'s are disjoint, so are $(B\cap A_i)$'s. Now since $\mu$ is a measure, we have, $\mu(\bigcup_{i=1}^{i=\infty}(B \cap A_i))  = \sum_{i=1}^{i=\infty}\mu(B \cap A_i) =
    \sum_{i=1}^{i=\infty} \lambda(A_i)$. Thus $\lambda(\bigcup_{i=1}^{i = \infty}A_i) = \sum_{i=1}^{i=\infty} \lambda(A_i)$. Hence $\lambda$ is a measure.

  \section*{Exercise 1.20}
    Let $A_1 \supset A_2 \supset ... \supset A_n$. This is equivalent to saying $(A_1 - A_1= \phi) \subset (A_1 - A_2) \subset (A_1 - A_3)... \subset(A_1 - A_n)$
    From the previous result, we have $\lim_{n \rightarrow \infty}\mu(A_1 - A_n) = \mu(\bigcup_{n=1}^{n=\infty}(A_1 - A_n)) = \mu(A_1 - \bigcap_{n=1}^{n=\infty}A_n)$ where we have used De Morgan's Law
    in the last step. We have already proved previously, the property of finite additivity of a measure. Therefore we have
    $\mu(A_1) - \lim_{n \rightarrow \infty}\mu(A_n) = \mu(A_1) - \mu(\bigcap_{n=1}^{n=\infty}A_n)$. Since $\mu(A_1) < \infty$, we can cancel it out from both sides to get the result.

  \section*{Exercise 2.10}
    To prove this result, we note that countable subadditivity of an outer-measure $\Rightarrow$ finite subadditivity. This can be seen by taking $A_i = \phi$ for $i > N$. Since $\mu^*(\phi)= 0$, we have
    $\mu^*(\bigcup_{i=1}^{i=N}A_i) \le \sum_{i=1}^{i=N}\mu^*(A_i)$ which follows from the definition of the outer-measure.
    \newline
    Now, we can write $B = (B \cap E) \cup (B \cap E^c)$. Therefore, using finite sub-additivity, we have $\mu^*(B) \le \mu^*(B \cap E) + \mu^*(B \cap E^c)$. Since the inequality in the other direction
    is already given, we can replace the inequality with an equality.

  \section*{Exercise 2.14}
    Let $\mathcal A = \{A: A$ is a countable union of intervals of the form $(a,b], (-\infty, b]$ and $(a, \infty)\}$. We first show that $\sigma(\mathcal A) \subset \sigma(\mathcal O)$. To see this,
    let $A \in \sigma(\mathcal A)$. We can write $(a,b] = \bigcap_{n=1}^{n=\infty}(a, b - 1/n), (- \infty, b] = \bigcap_{n=1}^{n=\infty}(-\infty, b- 1/n)$. Thus $A$ can be written as a countable union of
    intervals of the form $\bigcap_{n=1}^{n=\infty}(a, b-1/n), \bigcap_{n=1}^{n=\infty}(-\infty, b-1/n), (a, \infty)$. By the property of a $\sigma$- algebras, each of these terms, being countable intersections of
    open intervals, belong to $\sigma(\mathcal O)$. Thus the countable union of these terms also belongs to $\sigma(\mathcal O)$. Thus $\sigma(\mathcal A) \subset \sigma(\mathcal O)$.
    \newline
    Now we show that $\sigma(\mathcal O) \subset \sigma(\mathcal A)$. To see this, let $A \in \sigma(\mathcal O)$. Thus $A$ is an open interval. Let $A = (a,b)$. We can write $A = (a,b) = \bigcup_{n=1}^{n=\infty}(a, b-1/n]$. Similarly,
    any interval of the form $(-\infty, b)$ can be written as $\bigcup_{n=1}^{n=\infty}(-\infty, b-1/n]$ and any interval of the form $(a, \infty)$ can be written as $\bigcup_{n=1}^{n=\infty}[a-1/n, \infty)$.
    Note that each of the terms is a countable union of sets that $\in \mathcal A$ which $\Rightarrow$ that they $\in \sigma(\mathcal A)$. Thus any countable union of open sets also $\in \sigma(\mathcal A)$.
    Thus $\sigma(\mathcal O) \subset \sigma(\mathcal A)$.
    \newline
    We thus have $\sigma(\mathcal A) = \sigma(\mathcal O)$. By Caratheodory's Theorem, $\mathcal B(\mathbb R) \subset \mathcal M$.

  \section*{Exercise 3.1}
    Let $X \subset \mathbb R$ be a countable set. Let $x_1, x_2, x_3 ...$ be the elements of $X$. For every $\epsilon > 0$, define, $A_n = (x_n - \frac{\epsilon}{2^{n+2}}, x_n + \frac{\epsilon}{2^{n+2}}) \forall n \in \mathbb N$.
    Let $\mu$ denote the Lebesgue Measure. Therefore $\mu(\bigcup_{n=1}^{n=\infty})A_n = \sum_{n=1}^{n=\infty}\frac{\epsilon}{2^{n+1}}$. Summing the terms of the Geometric Progression on the RHS, we get
    $\mu(\bigcup_{n=1}^{n=\infty})A_n = \epsilon /2$. Since $\epsilon$ is arbitrary, we get $\mu(\bigcup_{n=1}^{n=\infty})A_n = 0$. Now each $x_n \in X$ also implies $x_n \in A_n$ as $A_n$ has been defined in a manner that includes $x_n$.
    Thus $X \subset \bigcup_{n=1}^{n=\infty}A_n$. By the monotonicity property, $\mu(X) \le \mu(\bigcup_{n=1}^{n=\infty}A_n) = 0$. Thus $\mu(X) = 0$ since the range of $\mu$ is non-negative.



















\end{document}
